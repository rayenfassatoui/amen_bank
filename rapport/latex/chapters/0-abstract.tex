\chapter*{Résumé}
\addcontentsline{toc}{chapter}{Résumé}

Dans le cadre de la transformation digitale des processus bancaires, ce projet de fin d'études présente la conception et la réalisation d'un \textbf{Système de Gestion des Fonds Bancaires} pour Amen Bank. Face aux défis de la gestion manuelle des mouvements de fonds inter-agences (provisionnement et versement), qui engendre des risques opérationnels, des délais prolongés et une traçabilité limitée, nous avons développé une solution web complète et sécurisée.

Le système mis en place permet d'automatiser l'intégralité du cycle de traitement des demandes de fonds : de la soumission par les agences, en passant par la validation de la Caisse Centrale, l'assignment des équipes de sécurité, jusqu'au dispatch et à la réception finale. L'application implémente un workflow multiniveaux avec un contrôle d'accès basé sur les rôles (RBAC), garantissant la sécurité et la traçabilité de chaque opération.

Sur le plan technique, nous avons adopté une architecture full-stack moderne basée sur \textbf{Next.js 15}, avec \textbf{PostgreSQL} comme système de gestion de base de données, \textbf{Prisma ORM} pour la couche d'accès aux données, et \textbf{NextAuth.js} pour l'authentification sécurisée. Le développement a suivi la méthodologie \textbf{Agile/Scrum}, réparti en quatre sprints de deux à trois semaines chacun.

Le système offre également un tableau de bord analytique permettant aux administrateurs de suivre les indicateurs clés de performance (KPIs), les tendances et les volumes de transactions, facilitant ainsi la prise de décision stratégique.

\textbf{Mots-clés :} Gestion des fonds bancaires, Workflow automatisé, RBAC, Next.js, PostgreSQL, Prisma, Authentification sécurisée, Scrum, Traçabilité, Analytics

\vspace{1cm}

\chapter*{Abstract}
\addcontentsline{toc}{chapter}{Abstract}

As part of the digital transformation of banking processes, this end-of-studies project presents the design and implementation of a \textbf{Banking Funds Management System} for Amen Bank. Facing the challenges of manual management of inter-agency fund movements (provisioning and remittance), which generates operational risks, prolonged delays and limited traceability, we have developed a complete and secure web solution.

The implemented system automates the entire fund request processing cycle: from submission by agencies, through validation by the Central Cash Department, security team assignment, to dispatch and final reception. The application implements a multi-level workflow with Role-Based Access Control (RBAC), ensuring security and traceability of each operation.

On the technical side, we adopted a modern full-stack architecture based on \textbf{Next.js 15}, with \textbf{PostgreSQL} as the database management system, \textbf{Prisma ORM} for the data access layer, and \textbf{NextAuth.js} for secure authentication. Development followed the \textbf{Agile/Scrum} methodology, divided into four sprints of two to three weeks each.

The system also provides an analytical dashboard allowing administrators to monitor key performance indicators (KPIs), trends and transaction volumes, thus facilitating strategic decision-making.

\textbf{Keywords:} Banking funds management, Automated workflow, RBAC, Next.js, PostgreSQL, Prisma, Secure authentication, Scrum, Traceability, Analytics

\newpage
