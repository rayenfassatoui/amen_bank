\chapter{Sprint 2 - Module Gestion des Demandes}

\section{Introduction}

Le Sprint 2 se concentre sur le développement du module central du système : la gestion des demandes de fonds. Ce module permet aux agences de créer des demandes détaillées spécifiant les coupures nécessaires, et offre aux différents acteurs (Agence, Caisse Centrale) une vue complète sur l'état des demandes. Ce sprint constitue le cœur fonctionnel de l'application.

\section{Vue d'Ensemble}

Le module de gestion des demandes couvre l'intégralité du cycle de vie d'une demande, de sa création à sa consultation. Les fonctionnalités principales incluent :

\begin{itemize}[leftmargin=*]
    \item Création de demandes avec spécification détaillée des coupures et pièces
    \item Validation automatique des montants (correspondance total vs détails)
    \item Liste des demandes avec filtrage avancé (statut, date, agence, montant)
    \item Consultation des détails complets d'une demande
    \item Historique des actions effectuées sur chaque demande
\end{itemize}

\section{Objectif Principal}

L'objectif principal du Sprint 2 est de permettre aux agences de soumettre des demandes de fonds précises et détaillées, tout en offrant à la Caisse Centrale une visibilité complète sur toutes les demandes en attente de traitement. Le système doit garantir la cohérence des données et faciliter la prise de décision.

\section{User Stories Détaillées}

Le tableau \ref{tab:user-stories-sprint2} présente les user stories du Sprint 2 avec leurs critères d'acceptation et priorités.

\begin{table}[H]
\centering
\caption{User Stories - Sprint 2}
\label{tab:user-stories-sprint2}
\small
\begin{tabular}{@{}p{1cm}p{2.5cm}p{5cm}p{4cm}p{1.5cm}@{}}
\toprule
\textbf{ID} & \textbf{En tant que} & \textbf{Je veux} & \textbf{Critères d'acceptation} & \textbf{Priorité} \\ 
\midrule
US-01 & Agence & Créer une demande de provisionnement & Formulaire complet, validation montants, confirmation & Must \\
\midrule
US-02 & Agence & Spécifier les coupures détaillées & Ajouter plusieurs denominations, calcul automatique & Must \\
\midrule
US-03 & Agence & Consulter mes demandes avec filtres & Liste paginée, filtres statut/date & Must \\
\midrule
US-04 & Caisse Centrale & Voir toutes les demandes & Vue globale, tri agence/date/statut & Must \\
\midrule
US-05 & Agence & Télécharger un PDF de ma demande & Export PDF formaté avec logo & Should \\
\bottomrule
\end{tabular}
\end{table}

\section{Analyse et Spécifications}

\subsection{Diagramme de Cas d'Utilisation - Sprint 2}

La figure \ref{fig:use-case-sprint2} présente les cas d'utilisation du Sprint 2.

\begin{figure}[H]
    \centering
    \includegraphics[width=0.9\textwidth]{use-case-sprint2.png}
    \caption{Diagramme de cas d'utilisation - Sprint 2}
    \label{fig:use-case-sprint2}
\end{figure}

Les cas d'utilisation principaux sont :
\begin{itemize}[leftmargin=*]
    \item \textbf{Créer demande} : L'agence soumet une nouvelle demande avec détails
    \item \textbf{Ajouter denominations} : Spécification des coupures et quantités
    \item \textbf{Consulter mes demandes} : L'agence visualise ses demandes avec filtres
    \item \textbf{Voir toutes demandes} : La Caisse Centrale a une vue globale
    \item \textbf{Calculer montant total} : Le système valide la cohérence
\end{itemize}

\subsection{Raffinements Textuels - Scénario Nominal}

\subsubsection{Cas d'utilisation : Créer une Demande}

\textbf{Acteur Principal :} Utilisateur Agence

\textbf{Préconditions :}
\begin{itemize}[leftmargin=*]
    \item L'utilisateur est authentifié avec le rôle Agence
    \item L'utilisateur est rattaché à une agence spécifique
\end{itemize}

\textbf{Scénario nominal :}
\begin{enumerate}[leftmargin=*]
    \item L'utilisateur accède à la page "Créer une demande"
    \item Le système affiche un formulaire vide
    \item L'utilisateur sélectionne le type de mouvement (Provisionnement ou Versement)
    \item L'utilisateur saisit le montant total souhaité (ex: 500,000 DT)
    \item L'utilisateur saisit une description optionnelle
    \item L'utilisateur ajoute les détails des coupures :
    \begin{itemize}
        \item Sélectionne une coupure (100 DT)
        \item Saisit la quantité (3000 billets)
        \item Le système calcule le subtotal (300,000 DT)
        \item Répète pour d'autres coupures (50 DT × 4000 = 200,000 DT)
    \end{itemize}
    \item Le système calcule le montant total des denominations (500,000 DT)
    \item Le système vérifie que le total correspond au montant saisi
    \item L'utilisateur clique sur "Soumettre la demande"
    \item Le système valide les données (schéma Zod)
    \item Le système génère un numéro unique de demande (ex: REQ-2024-001)
    \item Le système sauvegarde la demande avec statut "SUBMITTED"
    \item Le système affiche une confirmation avec le numéro de demande
    \item Le système envoie une notification à la Caisse Centrale
\end{enumerate}

\textbf{Postconditions :}
\begin{itemize}[leftmargin=*]
    \item Une nouvelle demande est créée dans la base de données
    \item La demande est visible dans la liste des demandes de l'agence
    \item La Caisse Centrale peut consulter cette demande
\end{itemize}

\subsubsection{Scénario Alternatif : Montants Incohérents}

\textbf{Point de divergence :} Étape 8 du scénario nominal

\begin{enumerate}[leftmargin=*]
    \item L'utilisateur saisit un montant total de 1,000,000 DT
    \item L'utilisateur ajoute des denominations pour un total de 950,000 DT
    \item Le système détecte une différence de 50,000 DT
    \item L'utilisateur tente de soumettre le formulaire
    \item Le système affiche un message d'erreur : "Le montant total ne correspond pas à la somme des denominations (différence: 50,000 DT)"
    \item L'utilisateur corrige soit le montant total, soit les denominations
    \item Retour à l'étape 9 du scénario nominal
\end{enumerate}

\subsection{Diagramme de Classes - Entités Demandes}

La figure \ref{fig:class-sprint2} présente le modèle de classes étendu incluant les entités de demandes.

\begin{figure}[H]
    \centering
    \includegraphics[width=\textwidth]{class-sprint2.png}
    \caption{Diagramme de classes - Sprint 2}
    \label{fig:class-sprint2}
\end{figure}

Les nouvelles entités sont :

\begin{description}[leftmargin=*]
    \item[Request] : Représente une demande de fonds avec son type, montant, statut et agence émettrice
    
    \item[DenominationDetail] : Détaille une ligne de coupure (dénomination, quantité, subtotal)
\end{description}

Le modèle Prisma correspondant :

\begin{lstlisting}[language=SQL]
model Request {
  id              Int      @id @default(autoincrement())
  requestNumber   String   @unique
  type            String   // "PROVISIONING" | "REMITTANCE"
  agency          Agency   @relation(fields: [agencyId], references: [id])
  agencyId        Int
  totalAmount     Float
  description     String?
  status          String   @default("SUBMITTED")
  denominations   DenominationDetail[]
  createdBy       User     @relation(fields: [createdById], references: [id])
  createdById     Int
  createdAt       DateTime @default(now())
  updatedAt       DateTime @updatedAt
  
  @@index([requestNumber])
  @@index([status])
  @@index([agencyId])
}

model DenominationDetail {
  id           Int     @id @default(autoincrement())
  request      Request @relation(fields: [requestId], references: [id], onDelete: Cascade)
  requestId    Int
  denomination Float   // 100, 50, 20, 10, 5, 2, 1
  quantity     Int
  subtotal     Float   // denomination * quantity
  
  @@index([requestId])
}
\end{lstlisting}

\subsection{Diagramme de Séquence - Création Demande}

La figure \ref{fig:sequence-create-request} illustre le processus complet de création d'une demande.

\begin{figure}[H]
    \centering
    \includegraphics[width=\textwidth]{sequence-create-request.png}
    \caption{Diagramme de séquence - Création de demande}
    \label{fig:sequence-create-request}
\end{figure}

Le processus détaillé :

\begin{enumerate}[leftmargin=*]
    \item L'agence remplit le formulaire avec type, montant et denominations
    \item Le frontend valide les données côté client (Zod schema)
    \item Une requête POST est envoyée à \texttt{/api/requests}
    \item L'API vérifie l'authentification et le rôle (doit être Agence)
    \item L'API valide à nouveau les données côté serveur
    \item L'API vérifie que le montant total = somme des subtotals
    \item L'API génère un requestNumber unique (format: REQ-YYYY-XXX)
    \item L'API démarre une transaction Prisma
    \item L'API crée l'enregistrement Request
    \item L'API crée les enregistrements DenominationDetail associés
    \item L'API commit la transaction
    \item L'API retourne la demande créée avec son numéro
    \item Le frontend affiche un message de succès
    \item Le frontend redirige vers la page de détails de la demande
\end{enumerate}

\subsection{Diagramme d'Activités - Consultation Demandes}

La figure \ref{fig:activity-consult-requests} présente le flux de consultation des demandes avec filtres.

\begin{figure}[H]
    \centering
    \includegraphics[width=0.75\textwidth]{activity-consult-requests.png}
    \caption{Diagramme d'activités - Consultation des demandes}
    \label{fig:activity-consult-requests}
\end{figure}

Le flux décisionnel inclut :

\begin{itemize}[leftmargin=*]
    \item Chargement initial des demandes selon le rôle (Agence = ses demandes, Caisse Centrale = toutes)
    \item Application optionnelle de filtres (statut, date, montant)
    \item Recherche par numéro de demande
    \item Tri des résultats
    \item Pagination (10, 25 ou 50 par page)
    \item Navigation vers détails si l'utilisateur clique sur une demande
\end{itemize}

\section{Conception Détaillée}

\subsection{Architecture Composants Frontend}

La structure des composants pour le module demandes :

\begin{verbatim}
app/requests/
├── page.tsx                    # Liste des demandes
├── create/
│   └── page.tsx               # Création de demande
└── [id]/
    └── page.tsx               # Détails d'une demande

components/requests/
├── request-filters.tsx        # Filtres et recherche
├── denomination-input.tsx     # Input coupures dynamique
├── request-table.tsx          # Tableau des demandes
└── request-details.tsx        # Vue détails complète
\end{verbatim}

\subsection{Routes API}

\begin{description}[leftmargin=*]
    \item[\texttt{GET /api/requests}] : Liste des demandes avec filtres
    \begin{itemize}
        \item Query params: agencyId, status, startDate, endDate, page, limit
        \item Retourne: \texttt{\{ requests: Request[], total: number, page: number \}}
    \end{itemize}
    
    \item[\texttt{POST /api/requests}] : Création d'une demande
    \begin{itemize}
        \item Body: \texttt{\{ type, agencyId, totalAmount, description?, denominations[] \}}
        \item Validation: Schema Zod
        \item Retourne: Request créée avec requestNumber
    \end{itemize}
    
    \item[\texttt{GET /api/requests/[id]}] : Détails d'une demande
    \begin{itemize}
        \item Include: denominations, agency, createdBy
        \item Retourne: Request complète
    \end{itemize}
    
    \item[\texttt{PUT /api/requests/[id]}] : Modification (description uniquement si statut = SUBMITTED)
    
    \item[\texttt{GET /api/requests/[id]/pdf}] : Export PDF de la demande
\end{description}

\subsection{Schémas de Validation Zod}

\begin{lstlisting}[language=JavaScript]
import { z } from 'zod';

const DenominationSchema = z.object({
  denomination: z.enum([100, 50, 20, 10, 5, 2, 1]),
  quantity: z.number().int().min(1).max(100000),
});

const CreateRequestSchema = z.object({
  type: z.enum(['PROVISIONING', 'REMITTANCE']),
  totalAmount: z.number().min(100).max(10000000),
  description: z.string().max(500).optional(),
  agencyId: z.number().int().positive(),
  denominations: z.array(DenominationSchema).min(1),
}).refine((data) => {
  const calculatedTotal = data.denominations.reduce(
    (sum, d) => sum + (d.denomination * d.quantity),
    0
  );
  return calculatedTotal === data.totalAmount;
}, {
  message: "Le montant total ne correspond pas aux denominations",
});
\end{lstlisting}

\section{Réalisation - Implémentation}

\subsection{Composant Création Demande}

Le composant de création de demande (\texttt{app/requests/create/page.tsx}) implémente :

\subsubsection{Formulaire Principal}

\begin{itemize}[leftmargin=*]
    \item \textbf{Section Informations Générales}
    \begin{itemize}
        \item Radio buttons pour type (Provisionnement / Versement)
        \item Input numérique pour montant total
        \item Textarea pour description (optionnel)
    \end{itemize}
    
    \item \textbf{Section Détails Denominations}
    \begin{itemize}
        \item Liste dynamique de lignes denomination
        \item Chaque ligne : Select (coupure) + Input (quantité) + Display (subtotal)
        \item Bouton "Ajouter une ligne" pour nouvelles coupures
        \item Bouton "Supprimer" sur chaque ligne
        \item Calcul automatique du total en temps réel
    \end{itemize}
    
    \item \textbf{Section Validation}
    \begin{itemize}
        \item Affichage du montant calculé vs montant saisi
        \item Indicateur visuel (vert si match, rouge si différence)
        \item Message d'erreur si incohérence
    \end{itemize}
    
    \item \textbf{Actions}
    \begin{itemize}
        \item Bouton "Soumettre" (disabled si validation échoue)
        \item Bouton "Annuler" (retour à la liste)
        \item Loader pendant la soumission
    \end{itemize}
\end{itemize}

\subsubsection{Gestion de l'État Local}

Le composant utilise React hooks pour gérer l'état :

\begin{lstlisting}[language=JavaScript]
const [type, setType] = useState('PROVISIONING');
const [totalAmount, setTotalAmount] = useState(0);
const [description, setDescription] = useState('');
const [denominations, setDenominations] = useState([
  { denomination: 100, quantity: 0 }
]);

const calculatedTotal = denominations.reduce(
  (sum, d) => sum + (d.denomination * d.quantity),
  0
);

const isValid = calculatedTotal === totalAmount && totalAmount > 0;
\end{lstlisting}

\subsection{Composant Liste Demandes}

Le composant de liste (\texttt{app/requests/page.tsx}) offre :

\subsubsection{Tableau des Demandes}

Colonnes affichées :
\begin{itemize}[leftmargin=*]
    \item Numéro de demande (cliquable, lien vers détails)
    \item Type (badge "Provisionnement" ou "Versement")
    \item Agence (nom complet)
    \item Montant (formaté avec séparateurs de milliers)
    \item Statut (badge coloré selon l'état)
    \item Date de création (format JJ/MM/AAAA HH:mm)
    \item Actions (icône "Voir détails")
\end{itemize}

\subsubsection{Système de Filtrage}

\begin{itemize}[leftmargin=*]
    \item \textbf{Filtre par statut} : Dropdown multi-select (SUBMITTED, VALIDATED, REJECTED, etc.)
    \item \textbf{Filtre par date} : Date picker range (de - à)
    \item \textbf{Filtre par montant} : Input range (min - max)
    \item \textbf{Recherche} : Input text recherchant dans requestNumber
    \item \textbf{Bouton "Réinitialiser"} : Efface tous les filtres
\end{itemize}

\subsubsection{Pagination}

\begin{itemize}[leftmargin=*]
    \item Sélecteur de nombre par page (10, 25, 50)
    \item Boutons Précédent/Suivant
    \item Affichage "Résultats 1-10 sur 45"
    \item Navigation directe vers une page spécifique
\end{itemize}

\subsection{Composant Détails Demande}

Le composant de détails (\texttt{app/requests/[id]/page.tsx}) structure l'information en sections :

\subsubsection{Header}

\begin{itemize}[leftmargin=*]
    \item Numéro de demande (titre principal)
    \item Badge de statut (couleur selon l'état)
    \item Informations agence (nom, code, localisation)
\end{itemize}

\subsubsection{Section Informations Générales}

Tableau récapitulatif :
\begin{itemize}[leftmargin=*]
    \item Type de mouvement
    \item Montant total
    \item Date de création
    \item Créé par (nom utilisateur)
    \item Description (si présente)
\end{itemize}

\subsubsection{Section Détails Denominations}

Tableau des coupures :
\begin{itemize}[leftmargin=*]
    \item Colonne "Dénomination" (100 DT, 50 DT, etc.)
    \item Colonne "Quantité" (nombre de billets/pièces)
    \item Colonne "Subtotal" (dénomination × quantité)
    \item Ligne de total (somme des subtotals)
\end{itemize}

\subsubsection{Section Timeline / Historique}

Affichage chronologique des événements :
\begin{itemize}[leftmargin=*]
    \item Icône selon le type d'action
    \item Date et heure
    \item Type d'action (Création, Validation, Rejet, etc.)
    \item Acteur (utilisateur ayant effectué l'action)
    \item Détails supplémentaires (motif de rejet, équipe assignée)
\end{itemize}

\subsubsection{Actions Disponibles}

Selon le statut et le rôle :
\begin{itemize}[leftmargin=*]
    \item Bouton "Télécharger PDF" (toujours visible)
    \item Bouton "Modifier" (si statut = SUBMITTED et rôle = Agence)
    \item Actions de validation (Sprint 3)
\end{itemize}

\subsection{État Zustand - Request Store}

Le store Zustand centralise l'état des demandes :

\begin{lstlisting}[language=JavaScript]
import create from 'zustand';

interface RequestState {
  requests: Request[];
  filters: {
    status: string[];
    dateFrom: Date | null;
    dateTo: Date | null;
    minAmount: number | null;
    maxAmount: number | null;
    search: string;
  };
  pagination: {
    page: number;
    limit: number;
    total: number;
  };
  setRequests: (requests: Request[]) => void;
  setFilters: (filters: Partial<FilterState>) => void;
  setPagination: (pagination: Partial<PaginationState>) => void;
  addRequest: (request: Request) => void;
  updateRequest: (id: number, updates: Partial<Request>) => void;
}

export const useRequestStore = create<RequestState>((set) => ({
  requests: [],
  filters: {
    status: [],
    dateFrom: null,
    dateTo: null,
    minAmount: null,
    maxAmount: null,
    search: '',
  },
  pagination: { page: 1, limit: 25, total: 0 },
  setRequests: (requests) => set({ requests }),
  setFilters: (newFilters) => 
    set((state) => ({ 
      filters: { ...state.filters, ...newFilters },
      pagination: { ...state.pagination, page: 1 } // Reset page
    })),
  setPagination: (newPagination) =>
    set((state) => ({
      pagination: { ...state.pagination, ...newPagination }
    })),
  addRequest: (request) =>
    set((state) => ({
      requests: [request, ...state.requests]
    })),
  updateRequest: (id, updates) =>
    set((state) => ({
      requests: state.requests.map(r =>
        r.id === id ? { ...r, ...updates } : r
      )
    })),
}));
\end{lstlisting}

\section{Tests et Validation}

\subsection{Tests Fonctionnels}

\subsubsection{Création de Demandes}

\begin{itemize}[leftmargin=*]
    \item [\checkmark] Création avec montant et denominations valides réussit
    \item [\checkmark] Numéro de demande unique généré automatiquement
    \item [\checkmark] Montants incohérents provoquent erreur de validation
    \item [\checkmark] Denomination dupliquée refusée
    \item [\checkmark] Montant négatif ou zéro refusé
    \item [\checkmark] Message de succès affiché après création
\end{itemize}

\subsubsection{Consultation}

\begin{itemize}[leftmargin=*]
    \item [\checkmark] Agence voit uniquement ses demandes
    \item [\checkmark] Caisse Centrale voit toutes les demandes
    \item [\checkmark] Filtres par statut fonctionnent correctement
    \item [\checkmark] Filtres par date limitent résultats
    \item [\checkmark] Recherche par requestNumber trouve la demande
    \item [\checkmark] Pagination affiche le bon nombre de résultats
\end{itemize}

\subsubsection{Détails}

\begin{itemize}[leftmargin=*]
    \item [\checkmark] Toutes les informations affichées correctement
    \item [\checkmark] Denominations listées avec calculs exacts
    \item [\checkmark] Timeline vide pour demande nouvellement créée
\end{itemize}

\subsection{Tests de Performance}

\begin{itemize}[leftmargin=*]
    \item [\checkmark] Liste de 1000 demandes charge en < 2 secondes
    \item [\checkmark] Filtrage réactif (< 300ms)
    \item [\checkmark] Création de demande avec 7 denominations < 1 seconde
\end{itemize}

\subsection{Tests d'Intégration}

\begin{itemize}[leftmargin=*]
    \item [\checkmark] Transaction Prisma garantit atomicité (Request + DenominationDetails)
    \item [\checkmark] Rollback en cas d'erreur ne laisse pas d'orphelins
    \item [\checkmark] Contraintes d'intégrité référentielle respectées
\end{itemize}

\section{Revue de Sprint}

\subsection{Démonstration}

Démonstration au Product Owner :

\begin{enumerate}[leftmargin=*]
    \item Création d'une demande de provisionnement (500,000 DT)
    \item Ajout de denominations multiples avec calcul automatique
    \item Tentative de soumission avec montants incohérents (erreur affichée)
    \item Correction et soumission réussie
    \item Consultation de la liste avec différents filtres
    \item Navigation vers détails d'une demande
    \item Export PDF (aperçu)
\end{enumerate}

\subsection{Validation Product Owner}

Points validés :
\begin{itemize}[leftmargin=*]
    \item Ergonomie de création de demande (intuitive)
    \item Système de validation des montants (fiable)
    \item Filtrage et recherche (complets et performants)
    \item Présentation des détails (claire et structurée)
\end{itemize}

\subsection{Feedback}

Suggestions retenues :
\begin{itemize}[leftmargin=*]
    \item Ajouter indicateur visuel du statut dans la liste (badges colorés) [\checkmark\ Implémenté]
    \item Permettre tri des colonnes du tableau [\textit{Ajouté au backlog}]
    \item Ajouter statistiques en haut de liste (nombre par statut) [\textit{Sprint 4}]
\end{itemize}

\subsection{Métriques du Sprint}

\begin{itemize}[leftmargin=*]
    \item \textbf{Durée} : 3 semaines
    \item \textbf{User Stories complétées} : 5/5 (100\%)
    \item \textbf{Points de complexité} : 34/34
    \item \textbf{Bugs identifiés} : 5 (4 résolus, 1 mineur reporté)
    \item \textbf{Lignes de code} : ~2,500 (frontend + backend)
\end{itemize}

\section{Conclusion du Chapitre}

Le Sprint 2 a permis de développer le module central du système : la gestion des demandes de fonds. Les agences peuvent désormais soumettre des demandes détaillées avec spécification précise des coupures nécessaires. Le système de validation garantit la cohérence des données, évitant les erreurs de saisie.

La Caisse Centrale dispose d'une vue complète sur toutes les demandes soumises, avec des outils de filtrage et de recherche performants facilitant la priorisation et le traitement. L'interface intuitive et responsive permet une adoption rapide par les utilisateurs.

Le chapitre suivant décrit le Sprint 3, qui implémente le workflow de validation par la Caisse Centrale et d'assignment des équipes de sécurité par Tunisie Sécurité, complétant ainsi la phase de préparation avant le dispatch.

\newpage
